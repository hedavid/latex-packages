% !TeX TXS-program:compile = txs:///arara
% arara: lualatex: {shell: no, synctex: no, interaction: batchmode}

\documentclass{article}
\usepackage[margin=0.5in]{geometry}
\usepackage{jetbrainsmono-otf}
\usepackage{listings}
\usepackage{xcolor}
\setlength{\parindent}{0pt}

\newcommand\demotext{For \textsterling 45, almost anything can be found floating in fields.\\
	!`THE DAZED BROWN FOX QUICKLY GAVE 12345--67890 JUMPS!\\
	--- ?`But aren't Kafka's Schlo\ss{} and \AE sop's \OE uvres often na\"\i ve vis-\`a-vis the d\ae monic ph\oe nix's official r\^ole in fluffy souffl\'es?
}

\newcommand\samplettxt{oO08 iIlL1 g9qCGQ <=>}
\newcommand\samplett[1][]{#1\samplettxt}
\newcommand\samplettit[1][]{\textit{#1\samplettxt}}
\newcommand\samplettbf[1][]{\textbf{#1\samplettxt}}
\newcommand\samplettbfit[1][]{\textbf{\textit{#1\samplettxt}}}
\newcommand\sampletttbl[1][]{& \samplett[#1] &  \samplettit[#1] & \samplettbf[#1] & \samplettbfit[#1]}

\begin{document}

\part*{jetbrainsmono-otf (v0.1)}

\section{Usages}

With \lstinline[language={[latex]TeX},basicstyle=\ttfamily]|\usepackage{fontspec}| (so with \lstinline[language={[latex]TeX},basicstyle=\ttfamily]|XeTeX| or \lstinline[language={[latex]TeX},basicstyle=\ttfamily]|LuaLaTeX| compilation), you can use \texttt{JetBrainsMono} fonts, and (if needed) \textit{remove} ligature's features, in order to use (in fact) \texttt{JetBrainsMonoNL} fonts. Following \textsf{OpenType} fonts are supported :

\begin{lstlisting}[language={[latex]TeX},basicstyle=\footnotesize\ttfamily,commentstyle=\itshape\color{gray},keywordstyle=\color{magenta},tabsize=4,frame=single]
JetBrainsMono-Bold.otf						JetBrainsMono-BoldItalic.otf
JetBrainsMono-ExtraBold.otf					JetBrainsMono-ExtraBoldItalic.otf
JetBrainsMono-ExtraLight.otf				JetBrainsMono-ExtraLightItalic.otf
JetBrainsMono-Italic.otf					JetBrainsMono-Light.otf
JetBrainsMono-LightItalic.otf				JetBrainsMono-Medium.otf
JetBrainsMono-MediumItalic.otf				JetBrainsMono-Regular.otf
JetBrainsMono-SemiBold.otf					JetBrainsMono-SemiBoldItalic.otf
JetBrainsMono-Thin.otf						JetBrainsMono-ThinItalic.otf
\end{lstlisting}

Two version of config/macro are available :

\begin{tabular}{l}
	~~~\lstinline[language={[latex]TeX},basicstyle=\ttfamily]|jetbrainsmono(-xxxx)| or \lstinline[language={[latex]TeX},basicstyle=\ttfamily]|jetbrainscode(-xxxx)|
	\\
	~~~\lstinline[language={[latex]TeX},basicstyle=\ttfamily]|\jetbrainsmono(xxxx)| or \lstinline[language={[latex]TeX},basicstyle=\ttfamily]|\jetbrainscode(xxxx)| \\
\end{tabular}

\section{The default settings}

The \texttt{fontspec} config for the \textit{normal} version :

\begin{lstlisting}[language={[latex]TeX},basicstyle=\footnotesize\ttfamily,commentstyle=\itshape\color{gray},keywordstyle=\color{magenta},tabsize=4,frame=single]
\defaultfontfeatures[jetbrainsmono]           % or jetbrainscode
{Extension=.otf,
    UprightFont=JetBrainsMono-Regular,
    ItalicFont=JetBrainsMono-Italic,
    BoldFont=JetBrainsMono-Bold,
    BoldItalicFont=JetBrainsMono-BoldItalic,
    Contextuals=AlternateOff                   %for mono version
}
\end{lstlisting}

The \texttt{fontspec} config for the other versions are :

\begin{lstlisting}[language={[latex]TeX},basicstyle=\footnotesize\ttfamily,commentstyle=\itshape\color{gray},keywordstyle=\color{magenta},tabsize=4,frame=single]
\defaultfontfeatures[jetbrainsmono-medium]     % or jetbrainscode-medium
    {Extension=.otf,
    UprightFont=JetBrainsMono-Medium,
    ItalicFont=JetBrainsMono-MediumItalic,
    BoldFont=JetBrainsMono-ExtraBold,
    BoldItalicFont=JetBrainsMono-ExtraBoldItalic,
    Contextuals=AlternateOff                   %for mono version
}
\end{lstlisting}

\begin{lstlisting}[language={[latex]TeX},basicstyle=\footnotesize\ttfamily,commentstyle=\itshape\color{gray},keywordstyle=\color{magenta},tabsize=4,frame=single]
\defaultfontfeatures[jetbrainsmono-light]      % or jetbrainscode-light
    {Extension=.otf,
    UprightFont=JetBrainsMono-Light,
    ItalicFont=JetBrainsMono-LightItalic,
    BoldFont=JetBrainsMono-SemiBold,
    BoldItalicFont=JetBrainsMono-SemiBoldItalic,
    Contextuals=AlternateOff                   %for mono version
}
\end{lstlisting}

\begin{lstlisting}[language={[latex]TeX},basicstyle=\footnotesize\ttfamily,commentstyle=\itshape\color{gray},keywordstyle=\color{magenta},tabsize=4,frame=single]
\defaultfontfeatures[jetbrainsmono-extralight] % or jetbrainscode-extralight
    {Extension=.otf,
    UprightFont=JetBrainsMono-ExtraLight,
    ItalicFont=JetBrainsMono-ExtraLightItalic,
    BoldFont=JetBrainsMono-Medium,
    BoldItalicFont=JetBrainsMono-MediumItalic,
    Contextuals=AlternateOff                   %for mono version
}
\end{lstlisting}

\begin{lstlisting}[language={[latex]TeX},basicstyle=\footnotesize\ttfamily,commentstyle=\itshape\color{gray},keywordstyle=\color{magenta},tabsize=4,frame=single]
\defaultfontfeatures[jetbrainsmono-thin]       % or jetbrainscode-thin
    {Extension=.otf,
    UprightFont=JetBrainsMono-Thin,
    ItalicFont=JetBrainsMono-ThinItalic,
    BoldFont=JetBrainsMono-Regular,
    BoldItalicFont=JetBrainsMono-Italic,
    Contextuals=AlternateOff                   %for mono version
}
\end{lstlisting}

\subsection{With config files}

The idea is to propose \texttt{fontspec} config files to load correctly \texttt{JetBrainsMono} features.

\begin{lstlisting}[language={[latex]TeX},basicstyle=\footnotesize\ttfamily,commentstyle=\itshape\color{gray},keywordstyle=\color{magenta},tabsize=4,frame=single]
\usepackage{fontspec}
\setmonofont{jetbrainsmono}[options]               %version regular
\setmonofont{jetbrainsmono-medium}[options]        %version semilight
\setmonofont{jetbrainsmono-light}[options]         %version light
\setmonofont{jetbrainsmono-extralight}[options]    %version extralight
\setmonofont{jetbrainsmono-thin}[options]          %version thin
\end{lstlisting}

\subsection{With the package loading}

With \lstinline[language={[latex]TeX},basicstyle=\ttfamily]|\usepackage[scale=...]{jetbrainsmono-oft}|, \lstinline[language={[latex]TeX},basicstyle=\ttfamily]|fontspec| is loaded, and \lstinline[language={[latex]TeX},basicstyle=\ttfamily]|fontfamily| are defined :

\begin{lstlisting}[language={[latex]TeX},basicstyle=\footnotesize\ttfamily,commentstyle=\itshape\color{gray},keywordstyle=\color{magenta},tabsize=4,frame=single]
\newfontfamily\jetbrainsmono{jetbrainsmono}
\newfontfamily\jetbrainsmonomedium{jetbrainsmono-medium}
\newfontfamily\jetbrainsmonolight{jetbrainsmono-light}
\newfontfamily\jetbrainsmonoextralight{jetbrainsmono-extralight}
\newfontfamily\jetbrainsmonothin{jetbrainsmono-thin}
\end{lstlisting}

\section{Font Samples}

\subsection{Normal version}

\setmonofont{jetbrainsmono}[Scale=MatchLowercase]

\texttt{\demotext}\par\bigskip

\texttt{\textit{\demotext}}\par\bigskip

\texttt{\textbf{\demotext}}\par\bigskip

\texttt{\textbf{\textit{\demotext}}}\par

\subsection{Medium version}

\setmonofont{JetBrainsMonoMedium}[Scale=MatchLowercase]

\texttt{\demotext}\par\bigskip

\texttt{\itshape\demotext}\par\bigskip

\texttt{\bfseries\demotext}\par\bigskip

\texttt{\bfseries\itshape\demotext}\par

\subsection{Light version (Light - LightItalic - Regular - Italic)}

\setmonofont{jetbrainsmono-light}[Scale=MatchLowercase]

\texttt{\demotext}\par\bigskip

\texttt{\itshape\demotext}\par\bigskip

\texttt{\bfseries\demotext}\par\bigskip

\texttt{\bfseries\itshape\demotext}\par

\subsection{ExtraLight version (ExtraLight - ExtraLightItalic - SemiLightItalic - SemiLightItalic)}

\setmonofont{jetbrainsmono-extralight}[Scale=MatchLowercase]

\texttt{\demotext}\par\bigskip

\texttt{\itshape\demotext}\par\bigskip

\texttt{\bfseries\demotext}\par\bigskip

\texttt{\bfseries\itshape\demotext}\par

\subsection{Thin version}

\setmonofont{jetbrainsmono-thin}[Scale=MatchLowercase]

\texttt{\demotext}\par\bigskip

\texttt{\itshape\demotext}\par\bigskip

\texttt{\bfseries\demotext}\par\bigskip

\texttt{\bfseries\itshape\demotext}\par

\section{Simple code samples}

\setmonofont{CMU Typewriter Text}[Scale=MatchLowercase]

\noindent{\small \begin{tabular}{lllll}
	\hline
	Type & Normal & Italic & Bold & BoldItalic \\ \hline
	ttdefault \sampletttbl[\ttfamily] \\ \hline
	%mono
	JBmono \sampletttbl[\jetbrainsmono] \\ \hline
	JBmonomedium \sampletttbl[\jetbrainsmonomedium] \\ \hline
	JBmonolight  \sampletttbl[\jetbrainsmonolight] \\ \hline
	JBmonoextralight \sampletttbl[\jetbrainsmonoextralight] \\ \hline
	JBmonothin \sampletttbl[\jetbrainsmonothin] \\ \hline
	%code
	JBcode \sampletttbl[\jetbrainscode] \\ \hline
	JBcodemedium \sampletttbl[\jetbrainscodemedium] \\ \hline
	JBcodelight \sampletttbl[\jetbrainscodelight] \\ \hline
	JBcodeextralight \sampletttbl[\jetbrainscodeextralight] \\ \hline
	JBcodethin \sampletttbl[\jetbrainscodethin] \\ \hline
\end{tabular}}

\section{Algorithm samples, without ligatures}

\subsection{Normal version}

\begin{lstlisting}[language=python,basicstyle=\footnotesize\jetbrainsmono,commentstyle=\itshape\color{gray},keywordstyle=\bfseries\color{magenta},tabsize=4,frame=single]
def Fibonacci(n) :
  # Check if input is 0 then it will print incorrect input
  if n < 0 :
    print("Incorrect input")
  elif n == 0 :
    return 0
  elif n == 1 or n == 2 :
    return 1
  else :
    return Fibonacci(n-1) + Fibonacci(n-2)
\end{lstlisting}

\subsection{Medium version}

\begin{lstlisting}[language=python,basicstyle=\footnotesize\jetbrainsmonomedium,commentstyle=\itshape\color{gray},keywordstyle=\bfseries\color{magenta},tabsize=4,frame=single]
def Fibonacci(n) :
  # Check if input is 0 then it will print incorrect input
  if n < 0 :
    print("Incorrect input")
  elif n == 0 :
    return 0
  elif n == 1 or n == 2 :
    return 1
  else :
    return Fibonacci(n-1) + Fibonacci(n-2)
\end{lstlisting}

\subsection{Light version}

\begin{lstlisting}[language=python,basicstyle=\footnotesize\jetbrainsmonolight,commentstyle=\itshape\color{gray},keywordstyle=\bfseries\color{magenta},tabsize=4,frame=single]
def Fibonacci(n) :
  # Check if input is 0 then it will print incorrect input
  if n < 0 :
    print("Incorrect input")
  elif n == 0 :
    return 0
  elif n == 1 or n == 2 :
    return 1
  else :
    return Fibonacci(n-1) + Fibonacci(n-2)
\end{lstlisting}

\subsection{ExtraLight version}

\begin{lstlisting}[language=python,basicstyle=\footnotesize\jetbrainsmonoextralight,commentstyle=\itshape\color{gray},keywordstyle=\bfseries\color{magenta},tabsize=4,frame=single]
def Fibonacci(n) :
  # Check if input is 0 then it will print incorrect input
  if n < 0 :
    print("Incorrect input")
  elif n == 0 :
    return 0
  elif n == 1 or n == 2 :
    return 1
  else :
    return Fibonacci(n-1) + Fibonacci(n-2)
\end{lstlisting}

\subsection{Thin version}

\begin{lstlisting}[language=python,basicstyle=\footnotesize\jetbrainsmonothin,commentstyle=\itshape\color{gray},keywordstyle=\bfseries\color{magenta},tabsize=4,frame=single]
def Fibonacci(n) :
  # Check if input is 0 then it will print incorrect input
  if n < 0 :
    print("Incorrect input")
  elif n == 0 :
    return 0
  elif n == 1 or n == 2 :
    return 1
  else :
    return Fibonacci(n-1) + Fibonacci(n-2)
\end{lstlisting}

\pagebreak

\section{Algorithm code, with ligatures}

\setmonofont{CMU Typewriter Text}[Scale=MatchLowercase]

Regular version of the fonts, with ligatures enable, can be uses with \texttt{code} alias (\lstinline[language={[latex]TeX},basicstyle=\ttfamily]|\jetbrainscode|).

\makeatletter
\renewcommand*\verbatim@nolig@list{}
\makeatother

\begin{lstlisting}[language=python,basicstyle=\footnotesize\jetbrainscode,commentstyle=\itshape\color{gray},keywordstyle=\bfseries\color{magenta},tabsize=4,frame=single,columns=flexible,showstringspaces=false]
\lstset{
  language=python,
  basicstyle=\footnotesize\jetbrainscode,
  commentstyle=\itshape\color{gray},
  keywordstyle=\bfseries\color{magenta},
  tabsize=4,
  frame=single,
  columns=flexible,
  showstringspaces=false
}
\end{lstlisting}

\begin{lstlisting}[language=python,basicstyle=\footnotesize\jetbrainscode,commentstyle=\itshape\color{gray},keywordstyle=\bfseries\color{magenta},tabsize=4,frame=single,columns=flexible,showstringspaces=false]
const similar = "oO08 iIlL1 g9qCGQ"
const diacritics_etc = "â é ù ï ø ç à Ē Æ œ"

window.toggleFavorite = (alias) => {
  try {
    let favorites = JSON.parse(localStorage.getItem('favorites')) || []
    if (favorites.indexOf(alias) > -1) {
      favorites = favorites.filter((v) => {
        return v !== alias
      })
    } else {
      favorites.push(alias)
    }
    localStorage.setItem('favorites', JSON.stringify(Array.from(new Set(favorites))))
  } catch (err) {
    // eslint-disable-next-line no-console
    console.error('could not save favorite', err)
  }
  renderSelectList()
  return false
}
\end{lstlisting}

\begin{lstlisting}[language=python,basicstyle=\footnotesize\jetbrainscode,commentstyle=\itshape\color{gray},keywordstyle=\bfseries\color{magenta},tabsize=4,frame=single,columns=flexible,showstringspaces=false]
def Fibonacci(n) :
  # Check if input is 0 then it will print incorrect input
  if n < 0 :
    print("Incorrect input")
  elif n == 0 :
    return 0
  elif 1 <= n <= 2 :
    return 1
  else :
    return Fibonacci(n-1) + Fibonacci(n-2)
\end{lstlisting}

\pagebreak

\section{History}

\begin{verbatim}
v0.1 Initial version
\end{verbatim}

\end{document}
